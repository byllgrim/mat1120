\paragraph{Teori} \hfill \\
Potensmetoden gjelder $n\times n$ matriser A
med en \emph{Strengt dominant egenverdi}.
$$|\lambda_1| > |\lambda_2| \geq ... \geq |\lambda_n|$$

Vis ser på $\vecx$ skrevet som
$$\vecx = c_1\vecv_1 + ... + c_n\vecv_n$$
$$A^k\vecx = c_1(\lambda_1)^k\vecv_1 + ... + c_n(\lambda_n)^k\vecv_n$$
Vi deler på den største egenverdien
$$\frac{1}{(\lambda_1)^k} A^k \vecx
  = c_1\vecv_1
    + c_2\left(\frac{\lambda_2}{\lambda_1}\right)^k\vecv_2
    + ...
    + c_n\left(\frac{\lambda_n}{\lambda_1}\right)^k\vecv_n$$

Fordi $\lambda_1$ er størst får man
$$(\lambda_1)^{-k} A^k \vecx \to c_1\vecv_1,\quad k\to\infty$$

Altså har vi at $A^k\vecx$ går i ca samme retning som $\vecv_1$.



\paragraph{Algoritme} \hfill \\
\begin{enumerate}
  \item Vel initialvektor $\vecx_0$ med største komponent $1$.
  \item For $k = 0,1,...$
    \begin{enumerate}[label=\alph*]
      \item Beregn $A\vecx_k$
      \item La $\mu_k$ være komponenten i $A\vecx_k$ med størst abs.
      \item Beregn $\vecx_{k+1} = (1/\mu_k)A\vecx_k$
    \end{enumerate}
  \item For nesten alle $\vecx_0$ vil
        $\mu_k \to \lambda_1$ og $\vecx_k \to \vecv_1$
\end{enumerate}
