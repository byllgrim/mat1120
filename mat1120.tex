\documentclass{article}
\usepackage{style}

\begin{document}
  \title{MAT1120}
  \author{Robin A. T. Pedersen}
  \maketitle
  \tableofcontents

  \section{Forord}
    Dette er en oversikt over alle definisjoner, teoremer og lignende
fra læreboka i MAT1120.

NB!
Noensteder har jeg skrevet $c\in\mathbb{R}$,
men det kan hende at $\mathbb{C}$ hadde fungert like fint.
Lignende \emph{"feil"} kan finnes andre steder.

NB!
Noen av kapitlene er mangelfulle.
Jeg har selv skrevet observasjoner og metoder.

  \setcounter{section}{3} % TODO Pass på at dette stemmer
  % TODO fjern "Kpt.X - "
  \section{Kpt.4 - Vektorrom}
    \subsection{Vektor rom og underrom}
      \subsubsection{Definisjon - vektorrom}
        Et vektorrom er en ikketom mengde V.
Den består av såkalte \emph{vektorer}.
Disse vektorene må være beskrevet av 2 operasjoner:
Addisjon og skalarmultiplikasjon.

De to operasjonene defineres av følgende aksiomer:
La $\vecu,\vecv,\vecw \in V$
\begin{enumerate}
  \item $\vecu + \vecv \in V$
  \item $\vecu + \vecv = \vecv + \vecu$
  \item $(\vecu+\vecv)+\vecw = \vecu+(\vecv+\vecw)$
  \item $\exists \vecO \in V s.a. \vecu+\vecO=\vecu$
  \item $\forall \vecu \in V,\ \exists -\vecu \in V s.a. \vecu+(-\vecu)=\vecO$
  \item $c\vecu \in V,\ c\in\mathbb{R}$
  \item $c(\vecu+\vecv) = c\vecu+c\vecv$
  \item $(c+d)\vecu = c\vecu+d\vecu$
  \item $c(d\vecu) = (cd)\vecu$
  \item $1\vecu = \vecu$
\end{enumerate}

      \subsubsection{Definisjon - underrom}
        Et underrom H er en delmengde av V.
H er et underrom av V.

To egenskaper må være oppfylt:
\begin{enumerate}
  \item H er lukket under addisjon.
        $\vecu+\vecv\in H,\ \forall \vecu,\vecv\in H$
  \item H er lukket under skalarmultiplikasjon.
        $c\vecu\in H,\ \forall c\in\mathbb{R}$
\end{enumerate}

      \subsubsection{Teorem 1}
        Hvis $\vecv_1, ..., \vecv_p$ er i et vektorrom V,
så er $\Span{\vecv_1, ..., \vecv_p}$ et underrom av V.

    \subsection{Nullrom, kolonnerom og lineærtransformasjoner}
      \subsubsection{Definisjon - nullrom}
        Nullromet til en $m\times n$ matrise A,
er mengden av alle løsninger av $A\vecx = \vecO$.
$$\Nul{A} = \{ \vecx\ :\ \vecx\in\mathbb{R}^n,\ A\vecx=\vecO \}$$

      \subsubsection{Teorem 2}
        Nullrommet til A $m\times n$, er et underrom av $\mathbb{R}^n$.

Med andre ord:
$A\vecx=\vecO$ har $m$ homogene lineære ligninger, med $n$ ukjente.
Mengden av løsninger er et underrom av $\mathbb{R}^n$.

      \subsubsection{Definisjon - kolonnerom}
        Kolonnerommet til $m\times n$ matrisen A,
er mengden av alle lineærkombinasjoner av kolonnene i A.

$A = [\veca_1\quad ...\quad \veca_n]$
$$\Col{A} = \Span{\veca_1, ..., \veca_n}$$

      \subsubsection{Teorem 3}
        Kolonnerommet til A $m\times n$, er et underrom av $\mathbb{R}^m$.

Med andre ord: Kolonnene i A har $m$ elementer i hver vektor.
Kolonnerommet er alle lineærkombinasjoner av disse,
og har derfor $m$ elementer i hver vektor.

      \subsubsection{Definisjon - lineærtransformasjon}
        En lineærtransformasjon T fra et vektorrom V til et annet vektorrom W,
er en regel som gir hver $\vecx$ i V en unik vektor $T(\vecx)$ i W.

To egenskaper må oppfylles
\begin{enumerate}
  \item $T(\vecu+\vecv) = T(\vecu)+T(\vecv),\ \forall\ \vecu,\vecv\in V$
  \item $T(c\vecu) = cT(\vecu),\ \forall\ c\in\mathbb{R}^n$
\end{enumerate}

      \subsubsection{Begrep - kjerne (kernel)}
        Praktisk talt synonymt med nullrom.
    \subsection{Lineært uavhengige mengder: basiser}
      \subsubsection{Teorem 4}
        En mengde $\{ \vecv_1, ..., \vecv_p \}$ (minst 2 vektorer)
er lineært avhengig hvis (minst) en vektor kan skrives som
en lineærkombinasjon av de andre vektorene.

      \subsubsection{Definisjon - basis}
        La H være et underrom av vektorrommet V.
En mengde $\mathcal{B} = \{ \vecb_1, ..., \vecb_p \}$ i V,
er en basis for H hvis:
\begin{enumerate}
  \item $\mathcal{B}$ er lineært uavhengig
  \item underrommet utspent av $\mathcal{B}$ er det samme som H.
        Altså, $H = \Span{\vecb_1, ..., \vecb_p}$
\end{enumerate}

      \subsubsection{Teorem 5 - utspennende mengde teoremet}
        La $S = \{ \vecv_1, ..., \vecv_p \}$ være en mengde i V,
og la $H = \Span{\vecv_1, ..., \vecv_p}$.
\begin{enumerate}
  \item Hvis $\vecv_k$ er en lin.komb. av de andre vektorene,
        så kan man fjerne den fra mengden
        og den vil fremdeles utspenne H.
  \item Hvis $H \neq \{ \vecO \}$, så er en delmengde av $S$ en basis for $H$.
\end{enumerate}

      \subsubsection{Teorem 6}
        Pivotkolonnene til en matrise A, utgjør en basis for $\Col{A}$.

Man velger altså de kolonnene i A som er lineært uavhengige.

    \subsection{Koordinatsystemer}
      \subsubsection{Teorem 7 - unik representasjon teoremet}
        La $\mathcal{B} = \{ \vecb_1, ..., \vecb_n \}$
være en basis for et vektorrom V.

Da fins in unik mengde $c_1,...,c_n \in \mathbb{R}$ s.a.
$$\vecx = c_1\vecb_1 + ... + c_n\vecb_n,
  \quad \forall\ \vecx \in V$$

      \subsubsection{Definisjon - $\mathcal{B}$-koordinater}
        Hvis $\mathcal{B} = \{ \vecb_1, ..., \vecb_n \}$ er en basis for V,
og $\vecx \in V$.

Koordinatene til $\vecx$ relativt til $\mathcal{B}$,
er vekter $c_1, ..., c_n$ s.a.
$$\vecx = c_1\vecb_1 + ... + c_n\vecb_n$$

Med andre ord: $\mathcal{B}$-koordinatene til $\vecx
  = [\vecx]_\mathcal{B}
  = (c_1, .., c_n)$.

      \subsubsection{Begrep - koordinatskiftematrise}
        Koordinatskiftematrisen $P_\B$,
tar en vektor fra $\B$ til standardbasis i $\R$,
$$\vecx = P_\B [\vecx]_\B$$
Hvor $P_\B$ lages enkelt ved
$$P_\B = [\vecb_1\ \vecb_2\ ...\ \vecb_n]$$

      \subsubsection{Teorem 8}
        La $\Basis$ være en basis for vektorrommet V.
Da er koordinatavbildningen $\vecx \mapsto [\vecx]_\B$
\emph{en-til-en} lineærtransformasjon fra V \emph{på} $\R^n$.

      \subsubsection{Begrep - isomorfi}
        En isomorfi er en \emph{en-til-en} og \emph{på} lineærtransformasjon.

Altså: den dekker hele V og enhver $\vecx$ har en unik $T(\vecx)$.

    \subsection{Dimensjon av vektorrom}
      \subsubsection{Teorem 9}
        Hvis et vektorrom V har en basis $\Basis$,
så er alle mengder i V med fler enn $n$ vektorer lineært \emph{avhengig}.

      \subsubsection{Teorem 10}
        Hvis et vektorrom V har en basis med $n$ vektorer,
så må \emph{alle} basiser for V ha nøyaktig $n$ vektorer.

      \subsubsection{Definisjon - dimensjon}
        Hvis V er utspent av en endelig mengde,
så er V \emph{endelig-dimensjonalt}.
Dimensjonen til V, Dim V, er antall vektorer i en basis for V.

Hvis V \emph{ikke} er utspent av en endelig mengde,
så er V \emph{uendelig-dimensjonalt}.

Dimensjonen til nullvektorrommet $\{\vecO\}$ er null.

      \subsubsection{Teorem 11}
        La H være et underrom av et endelig-dimensjonalt vektorrom V.
Alle lineært uavhengige mengder i V kan utvides, hvis nødvendig,
til en basis for H.

H er også endelig-dimensjonalt.
$$\dim{H} \leq \dim{V}$$

      \subsubsection{Teorem 12 - basisteoremet}
        La V være et p-dimensjonalt vektorrom, $p\geq 1$.

Alle lin.uavh. mengder med nøyaktig p elementer i V, er en basis for V.

Alle mengder som spenner V med nøyaktig p elementer, er en basis for V.

      \subsubsection{Observasjon - DimNul og DimCol}
        Dimensjonen til $\Nul{A}$ er antall fri variable i $A\vecx = \vecO$.

Dimensjonen til $\Col{A}$ er antall pivot-kolonner i A.

    \subsection{Rang}
      \subsubsection{Definisjon - radrom}
        Radrommet til A, $\Row{A}$,
er mengden av alle lineærkonbinasjoner av radvektorene i A.

      \subsubsection{Teorem 13}
        A og B er radekvivalente hvis $\Row{A} = \Row{B}$.

Hvis B er på trappeform, så er ikkenull radene i B
en basis for både $\Row{A}$ og $\Row{B}$.

      \subsubsection{Definisjon - rang}
        Rangen til A er dimensjonen til kolonnerommet til A.
$$\Rank{A} = \Dim{\Col{A}}$$

      \subsubsection{Teorem 14 - rangteoremet}
        Kolonne-rang er det samme som rad-rang:
$$\Dim{\Col{A}} = \Dim{\Row{A}}$$

Rangen til A er lik antall pivotelementer i A.

Rangen til A oppfyller:
$$\Rank{A} + \Dim{\Nul{A}}$$

      \subsubsection{Teorem - invertibel matrise teoremet (fortsatt)}
        Med $A n\times n$, så er følgende påstander ekvivalente
\begin{enumerate}
  \item A er invertibel.
  \item Kolonnene i A er en basis for $\R^n$.
  \item $\Col{A} = \R^n$.
  \item $\Dim{\Col{A}} = n$.
  \item $\Rank{A} = n$.
  \item $\Nul{A} = {\vecO}$.
  \item $\Dim{\Nul{A}} = 0$.
\end{enumerate}

    \subsection{Basisskifte}
      \subsubsection{Teorem 15}
        La $\Basis$ og $\Casis$ være basiser for $V$.

Da finnes en unik $n\times n$ matrise $\underset{\C\gets\B}{P}$ s.a.
$$[\vecx]_\C = \underset{\C\gets\B}{P} [\vecx]_\B$$

Hvor
$$\underset{\C\gets\B}{P} = [\ [\vecb_1]_\C\quad ...\quad [\vecb_n]_\C\ ]$$

      \subsubsection{Begrep - koordinatskiftematrise}
        Matrisen $\underset{\C\gets\B}{P}$ kalles for
koordinatskiftematrisen fra $\B$ til $\C$.

      \subsubsection{Observasjon - Invers av koord.skiftematr.}
        $$\left( \underset{\C\gets\B}{P} \right)^{-1} = \underset{\B\gets\C}{P}$$

    \subsection{Ikke eksamensrelevant}
      Ikke eksamensrelevant.
    \subsection{Anvendelser til Markovkjeder}
      \subsubsection{Begrep - sannsynlighetsvektor}
        En sannsynlighetsvektor: har ikkenegative elementer, og summerer til 1.

      \subsubsection{Begrep - stokastisk matrise}
        En stokastisk matrise:
en kvadratisk matrise med sannsynlighetsvektorer som kolonner.

      \subsubsection{Begrep - markovkjede}
        En markovkjede er en følge av sannsynlighetsvektorer,
sammen med en stokastisk matrise P s.a.
$$\vecx_{k+1} = P\vecx_k,\quad k=0,1,2,...$$

      \subsubsection{Begrep - tilstandsvektor}
        Et element $\vecx_k$ i markovkjeden.

      \subsubsection{Begrep - ekvilibriumsvektor}
        Tilstandsvektorene i markovkjeden forandres for hver iterasjon,
men hvis man finner en vektor som ikke endres, er det en ekvilibriumsvektor.
$$P\vecq = \vecq$$

Alle stokastiske matriser har en ekvilibriumsvektor.

      \subsubsection{Begrep - regulæritet}
        Hvis en potens av stokastisk matrise $P^k$ kun inneholder positive elementer,
så er den regulær.

      \subsubsection{Teorem 18}
        Hvis P er regulær og stokastisk,
så vil markovkjeden konvergere mot den unike ekvilibriumsmatrisen.
$$\{\vecx_k\}\to\vecq \qquad \text{når}\ k\to\infty$$

  \section{Kpt.5 - Egenverdier og Egenvektorer}
    \subsection{Egenvektor og egenverdier}
      \subsubsection{Definisjon - egenvektor og egenverdi}
        En \emph{egenvektor} til matrisen A,
er en ikkenul vektor $\vecx$ s.a.
$$A\vecx = \lambda\vecx$$
Hvor $\lambda$ er en egenverdi til A hvis det finnes en ikketriviell løsning.

      \subsubsection{Begrep - egenrom}
        $$A\vecx = \lambda\vecx$$
$$A\vecx - \lambda\vecx = 0$$
$$(A - \lambda I)\vecx = 0$$
Mengden av alle løsninger kalles \emph{egenrommet} til A for $\lambda$.

      \subsubsection{Teorem 1}
        Egenverdiene til en triangulær matrise er elementene langs hoveddiagonalen.

      \subsubsection{Teorem 2}
        Hvis $\vecv_1, ..., \vecv_r$ er egenvektorer til $\lambda_1, ..., \lambda_r$,
for en matrise A,
så er mengden $\{ \vecv_1, ..., \vecv_r \}$ lineært uavhengig.

      \subsubsection{}
        TODO Egenvektorer og differensligninger
    \subsection{Den karakteristisk ligningen}
      \subsubsection{Teorem - IMT fortsatt}
        Invertibel matrise teoremet:

Følgende er ekvivalent:
\begin{enumerate}
  \item A er invertibel.
  \item 0 er ikke en egenverdi til A.
  \item $\Det{A} \neq 0$.
\end{enumerate}

For $A\ 3\times 3$, så er $|\Det{A}|$ volumet utspent av kolonnene.
Hvis volumet er null, så har kolonnene kollapset inn i hverandre og er
lineært \emph{avhengige}.

      \subsubsection{Teorem 3 - egenskaper til determinanter}
        La A og B være $n\times n$ matriser.
Da gjelder følgende:
\begin{enumerate}
  \item A er invertibel $\iff$ $\Det{A} \neq 0$.
  \item $\Det{AB} = (\Det{A})(\Det{B})$.
  \item $\Det{A^T} = \Det{A}$.
  \item A triangulær $\implies$ $\Det{A} =$ produktet av diagonalelementene.
  \begin{enumerate}[label=\alph*]
    \item Radmultippel endrer ikke determinanten.
    \item Radbytte endrer determinantens fortegn.
    \item Radskalering, skalerer determinanten.
  \end{enumerate}
\end{enumerate}

      \subsubsection{Begrep - karakteristisk ligning}
        $\lambda$ er en egenverdi for A $\iff \Det{A - \lambda I} = 0$.

      \subsubsection{Teorem 4}
        Hvis Matrisene A,B $n\times n$ har samme karakteristiske polynom,
altså samme egenverdier med lik multiplisitet,
så er de similære.
$$B = P^{-1}AP$$

      \subsubsection{}
        TODO Anvendelse til dynamiske systemer
    \subsection{Diagonalisering}
      \subsubsection{Teorem 5 - diagonaliseringsteoremet}
        A $n\times n$ er diagonaliserbar $\iff$ A har n lin.uavh. egenvektorer.
$$A=PDP^{-1} \iff
\text{P = n lin.uavh. egenvek. til A,
      og D = diag(egenverdiene)}$$

Altså: A diagonaliserbar hhvis nok egenvek. til en basis for $\R^n$.

      \subsubsection{Metode - diagonalisering}
        \begin{enumerate}
  \item Finn egenverdiene til A.
  \item Finn lineært uavhengige egenvektorer.
  \item Lag $P = [\ \vecv_1\ \vecv_2 \quad ... \quad \vecv_n\ ]$.
  \item Lag $D = \text{diag}(\lambda_1, \lambda_2, ..., \lambda_n)$.
\end{enumerate}

      \subsubsection{Teorem 6}
        En $n\times n$ matrise med $n$ distinkte egenverdier er diagonaliserbar.

Merk: Det trenger ikke finnes $n$ distinkte egen\emph{verdier}.

      \subsubsection{Teorem 7}
        La A være en $n\times n$ matrise,
med distinkte egenverdier $\lambda_1, ..., \lambda_p$.
Da gjelder følgende:
\begin{enumerate}
  \item Dimensjonen til en egenverdis egenrom,
        er mindre eller lik multiplisiteten.
  \item A diagonaliserbar $\iff$ sum av dimensjon til egenrommene er lik n.
        Det er kun tilfellet hhvis:
        \begin{enumerate}[label=\alph*]
          \item Karakteristisk polynom kan faktoriseres til lineære faktorer.
          \item Dimensjonen til egenrom er lik tilsvarende multiplisitet.
        \end{enumerate}
  \item Hvis A er diag.bar og $\B_k$ er basis for egenrom til $\lambda_k$:
        Så er $\B_1, ..., \B_p$ egenvektorbasis for $\R^n$.
\end{enumerate}

    \subsection{Egenvektorer og lineærtransformasjoner}
      \subsubsection{Metode - relativ transformasjonsmatrise}
        La V være n-dimensjonalt vektorrom, W et m-dimensjonalt vektorrom,
$\B$ basis for V, og $\C$ basis for W,
og $T: V \to W$.

Da kan man finne en matrise M s.a.
$$[T(\vecx)]_\C = M [\vecx]_\B$$

ved at
$$M = [\ [T(\vecb_1)]_\C \quad [T(\vecb_2)]_\C \quad ... \quad [T(\vecb_n)]_\C$$

M kalles for: Matrisen til T relativ til basisene $\B$ og $\C$.

      \subsubsection{Metode - lin.transformasjon fra V til V}
        Matrisen M for T relativ til $\B$ kalles her for $[T]_\B$.

$$[T(\vecx)]_\B = M [\vecx]_\B$$
$$[T(\vecx)]_\B = [T]_\B [\vecx]_\B$$

$[T]_\B$ er $\B$-matrisen til T.

      \subsubsection{Teorem 8 - Diagonal matrise representasjon}
        Hvis $A = PDP^{-1}$, og $\B$ formes fra kolonnene i P
til å være en basis for $\R^n$.

Da er D $\B$-matrisen til $\vecx \mapsto A\vecx$.
$$D = [T]_\B$$

      \subsubsection{}
        TODO Similaritet av matriserepresentasjoner
    \subsection{Komplekse egenverdier}
      \subsubsection{Teorem 9}
        A $2\times 2$ reell matrise,
med kompleks egenverdi $\lambda = a - bi$, $b\neq 0$,
og tilhørende $\vecv \in \mathbb{C}^2$.

Da er
$$A = PCP^{-1}, \quad P=[\re{\vecv}\quad\im{\vecv}], \quad C = 
  \begin{bmatrix}
    a & -b \\
    b & a
  \end{bmatrix}
$$

      \subsubsection{Metode - Spesielt tilfelle}
        Hvis $C = \begin{bmatrix}a&-b\\b&a\end{bmatrix}$,
hvor $a,b\in\R$ og $a,b\neq 0$.
Så vil egenverdine til $C$ være $\lambda = a \pm bi$.

La $r = |\lambda| = \sqrt{a^2 + b^2}$.
Da er
$$C = r\begin{bmatrix}
         a/r & -b/r \\
         b/r & a/r
       \end{bmatrix}
  = \begin{bmatrix} r & 0 \\ 0 & r \end{bmatrix}
    \begin{bmatrix} \cos{\phi} & -\sin{\phi} \\
                    \sin{\phi} & \cos{\phi} \end{bmatrix}
$$

    \subsection{Diskrete dynamiske systemer}
      \subsubsection{Metode - følger}
        For systemer av typen $\vecx_{k+1} = A\vecx_k$:

Anta at A er diag.bar med n lin.uavh. egenvek. $\vecv_1, ..., \vecv_n$
med tilsvarende (ordnede) egenverdier $|\lambda_1| \geq ... \geq |\lambda_n|$.

Initialvektor egenvektordekomposisjon:
$$\vecx_0 = c_1\vecv_1 + ... + c_n\vecv_n$$

Iterasjon:
$$\vecx_1 = A\vecx_0
  = ...
  = c_1\lambda_1\vecv_1 + ... + c_n\lambda_n\vecv_n$$

Generelt:
$$\vecx_k = c_1(\lambda_1)^k\vecv_1 + ... + c_n(\lambda_n)^k\vecv_n$$

      \subsubsection{Observasjon - origos natur}
        \begin{enumerate}
  \item Alle $|\lambda| < 1 \implies$ origo er attraktor.
  \item Alle $|\lambda| > 1 \implies$ origo er frastøter.
  \item Minst én $|\lambda| > 1$ og én $|\lambda|< 1$
        $\implies$ origo er sadelpunkt.
\end{enumerate}

      \subsubsection{}
        TODO bytte av variabel, komplekse egenverdier
    \subsection{Anvendelser til differensialligninger}
      \subsubsection{Repetisjon - Diffligninger}
        La $x(t)$ være en funksjon og $a \in \R$.
Gitt ligningen
$$x'(t) = a\cdot x(t)$$
Så er
$$x(t) = c\cdot e^{a\cdot t}$$

      \subsubsection{Metode - initialverdiproblem}
        Gitt egenverdier $\lambda_1,\lambda_2$,
og egenvektorer $\vecv_1, \vecv_2$,
og initialverdi $\vecx(0)$:
Løs ligningen $\vecx'(t) = A\vecx$.

\paragraph{Løsning} \hfill \\
$$\vecx(t) = c_1\vecv_1 e^{\lambda_1t} + c_2\vecv_2 e^{\lambda_2t}$$

      \subsubsection{Observasjon - frastøter, sadel, attraktor}
        Hvis egenverdiene er positive, så er origo en frastøter.

Hvis egenverdiene er negative, så er origo en attraktor.

Hvis egenverdiene er blandet, så er origo et sadelpunkt.

      \subsubsection{Metode - avkobling av dynamiske systemer}
        Når vi har $\vecx' = A\vecx$, med A diagonaliserbar $A = PDP^{-1}$.
Så kan vi gjøre et variabelskifte $\vecy = P^{-1}\vecx$.

$$x' = \frac{d}{dt}(P\vecy)
     = A (p\vecy)
     = (PDP^{-1})P\vecy
     = PD\vecy$$

Venstremultipliser med $P^{-1}$ og få
$$\vecy' = D\vecy$$
Det er mye enklere å løse.

      \subsubsection{Komplekse egenverdier}
        Hvis matrisen A har komplekse egenverdier $\lambda$ og egenvektorer $\vecv$,
så kan vi finne generelle løsninger.



\paragraph{Kompleks generell løsning} \hfill \\
På vanlig vis:
$$\vecx(t) = c_1\vecv_1e^{\lambda_1t} + c_2\vecv_2e^{\lambda_2t}$$



\paragraph{Reell generell løsning} \hfill \\
$$\vecx(t) = c_1\vecy_1(t) + c_2\vecy_2(t)$$
$$\vecy_1 = \re\vecx_1
          = ([\re\vecv]\cos{bt} - [\im\vecv]\sin{bt})e^{at}$$
$$\vecy_2 = \re\vecx_2
          = ([\re\vecv]\sin{bt} + [\im\vecv]\cos{bt})e^{at}$$

hvor $\lambda_1 = a + bi$ og $\vecx_1 = \vecv_1e^{\lambda_1t}$.

    \subsection{Iterative estimater for egenverdier}
      \subsubsection{Metode - potensmetoden}
        \paragraph{Teori} \hfill \\
Potensmetoden gjelder $n\times n$ matriser A
med en \emph{Strengt dominant egenverdi}.
$$|\lambda_1| > |\lambda_2| \geq ... \geq |\lambda_n|$$

Vis ser på $\vecx$ skrevet som
$$\vecx = c_1\vecv_1 + ... + c_n\vecv_n$$
$$A^k\vecx = c_1(\lambda_1)^k\vecv_1 + ... + c_n(\lambda_n)^k\vecv_n$$
Vi deler på den største egenverdien
$$\frac{1}{(\lambda_1)^k} A^k \vecx
  = c_1\vecv_1
    + c_2\left(\frac{\lambda_2}{\lambda_1}\right)^k\vecv_2
    + ...
    + c_n\left(\frac{\lambda_n}{\lambda_1}\right)^k\vecv_n$$

Fordi $\lambda_1$ er størst får man
$$(\lambda_1)^{-k} A^k \vecx \to c_1\vecv_1,\quad k\to\infty$$

Altså har vi at $A^k\vecx$ går i ca samme retning som $\vecv_1$.



\paragraph{Algoritme} \hfill \\
\begin{enumerate}
  \item Vel initialvektor $\vecx_0$ med største komponent $1$.
  \item For $k = 0,1,...$
    \begin{enumerate}[label=\alph*]
      \item Beregn $A\vecx_k$
      \item La $\mu_k$ være komponenten i $A\vecx_k$ med størst abs.
      \item Beregn $\vecx_{k+1} = (1/\mu_k)A\vecx_k$
    \end{enumerate}
  \item For nesten alle $\vecx_0$ vil
        $\mu_k \to \lambda_1$ og $\vecx_k \to \vecv_1$
\end{enumerate}

      \subsubsection{Metode - invers potensmetode}
        \paragraph{Teori} \hfill \\
Metoden tilnærmer hvilkensomhelst egenverdi,
gitt at initialgjetning $\alpha$ er nærme nok $\lambda$.

La $B = (A - \alpha I)^{-1}$ og bruk potensmetoden på B.

Egenverdiene til A er $\lambda_1, ..., \lambda_n$,
og egenverdiene til B er
$\frac{1}{\lambda_1 - \alpha}, ..., \frac{1}{\lambda_n - \alpha}$.

Egenverdiene til A vil ligge innenfor $[-|\lambda_1|, |\lambda_1|]$.



\paragraph{Algoritme} \hfill \\
\begin{enumerate}
  \item Velg initialgjetning $\alpha$ nærme $\lambda$
  \item Velg initialvektor $\vecx_0$ med største komponent 1.
  \item For $k = 0,1,...$
    \begin{enumerate}[label=\alph*]
      \item Beregn $(A-\alpha I)\vecy_k = \vecx_k$
      \item La $\mu_k$ være komponent i $\vecy_k$ med størst abs.
      \item Beregn $v_k = \alpha + (1/\mu_k)$
      \item Beregn $\vecx_{k+1} = (1/\mu_k)\vecy_k$
    \end{enumerate}
  \item $v_k \to \lambda$ og $\vecx_k \to \vecv$
\end{enumerate}

  \section{Kpt.6 - Ortogonalitet og Minstekvadrater}
    \subsection{Indre produkt, lengde og ortogonalitet}
      \subsubsection{Teorem 1}
        For $\vecu,\vecv,\vecw \in \R^n$, og $c \in \R$ så gjelder
\begin{enumerate}
  \item $\vecu\cdot\vecv = \vecv\cdot\vecu$
  \item $(\vecu+\vecv)\cdot\vecw = \vecu\cdot\vecw + \vecv\cdot\vecw$
  \item $(c\vecu)\cdot\vecv = c(\vecu\cdot\vecv) = \vecu\cdot(c\vecv)$
  \item $\vecu\cdot\vecu \geq 0, \qquad \vecu\cdot\vecu=0 \iff \vecu=0$
\end{enumerate}

      \subsubsection{Definisjon - norm}
        Lengden (normen) til en vektor er en skalar gitt ved
$$||\vecv|| = \sqrt{\vecv\cdot\vecv}$$

      \subsubsection{Definisjon - distanse}
        Avstanden mellom to vektorer, skrevet $\text{dist}(\vecu,\vecv)$,
er lengden av vektoren $\vecu-\vecv$.
$$\text{dist}(\vecu,\vecv) = ||\vecu-\vecv||$$

      \subsubsection{Definisjon - ortogonalitet}
        To vektorer i $R^n$ er ortogonale hvis
$$\vecu\cdot\vecv = 0$$

      \subsubsection{Teorem 2 - pytagoras teorem}
        To vektorer er ortogonale
$\iff ||\vecu+\vecv||^2 = ||\vecu||^2 + ||\vecv||^2$

      \subsubsection{Begrep - ortogonalt komplement}
        For et underrom W, finnes det vektorer som står normalt på dette underrommet.
Spennet av et utvalg slike vektorer utgjør da $W^\perp$.

F.eks. i $R^3$ kan $\vecu,\vecv$ utspenne W, og en vektor som står normalt
på $\vecu-\vecv$ utstpenner $W^\perp$.

      \subsubsection{Teorem 3}
        Ortogonale komplement
$$(\Row{A})^\perp = \Nul{A}$$
$$(\Col{A})^\perp = \Nul{A^T}$$

    \subsection{Ortogonale mengder}
      \subsubsection{Teorem 4}
        Hvis $S = \{ \vecu_1, ..., \vecu_p\}$ er en ortogonal mengde,
hvor $\vecu_i \neq 0 \in \R^n$,
så er S lineært uavhengig og derfor en basis for underrommet $\Span{S}$.

      \subsubsection{Teorem 5}
        La $\{ \vecu_1, ..., \vecu_p \}$ være ortogonal basis for W i $\R^n$.
$$\forall\ \vecy\in W, \qquad \vecy = c_1\vecu_1 + ... + c_p\vecu_p$$
Vektene bestemmes ved
$$c_j = \frac{\vecy\cdot\vecu_j}{\vecu_j\cdot\vecu_j}, \qquad j=1,...,p$$

Altså er $\vecy$ summen av komponenten langs hver av
de ortogonale vektorene i basisen.

      \subsubsection{Metode - ortogonal projeksjon}
        Gitt en basis $\{ \vecu_1, ..., \vecu_p \}$ for et underrom W.
En vektor $\vecy\in\R^n$ kan dekomponeres til en sum av vektorkomponent
langs $\vecu_i$ pluss en vektor $\vecz$ ortogonal på
en utvidelse av basisen til $\R^n$.
$$\vecy = \hat{\vecy} + \vecz$$

$\hat{\vecy}$ er projeksjonen av $\vecy$ langs W.
$$\hat{\vecy} = \text{proj}_W\vecy
  = \frac{\vecy\cdot\vecu_1}{\vecu_1\cdot\vecu_1}\vecu_1
    + \frac{\vecy\cdot\vecu_2}{\vecu_2\cdot\vecu_2}\vecu_2
    + ...
    + \frac{\vecy\cdot\vecu_p}{\vecu_p\cdot\vecu_p}\vecu_p$$

        TODO feil seksjon?
      \subsubsection{Teorem 6}
        U $m\times n$ har ortonormale kolonner $\iff U^TU = I$

      \subsubsection{Teorem 7}
        La U $m\times n$ med ortonormale kolonner, $\vecx,\vecy\in\R^n$.
Da er
\begin{enumerate}
  \item $||U\vecx|| = ||\vecx||$
  \item $(U\vecx)\cdot(U\vecy) = \vecx\cdot\vecy$
  \item $(U\vecx)\cdot(U\vecy)=0 \iff \vecx\cdot\vecy=0$
\end{enumerate}

      \subsubsection{}
        TODO ortogonal basis def
    \subsection{Ortogonal projeksjon}
      \subsubsection{Teorem 8 - ortogonal dekomposisjon}
        La W være underrom av $\R^n$.
Da kan alle $\vecy\in\R^n$ skrives unikt som
$$\vecy = \hat{\vecy} + \vecz$$
Hvor $\hat{\vecy}\in W$ og $\vecz\in W^\perp$.

$$\hat{\vecy} = \frac{\vecy\cdot\vecu_1}{\vecu_1\cdot\vecu_1}\vecu_1
                + \frac{\vecy\cdot\vecu_2}{\vecu_2\cdot\vecu_2}\vecu_2
                + ...
                + \frac{\vecy\cdot\vecu_p}{\vecu_p\cdot\vecu_p}\vecu_p$$
for en basis $\{ \vecu_1, ..., \vecu_p \}$ for W.

$$\vecz = \vecy - \hat{\vecy}$$
$\hat{\vecy}$ er en ortogonal projeksjon på W,
og skrives $\text{proj}_W\vecy$.

      \subsubsection{Teorem 9 - beste approksimasjon}
        Hvis W underrom av $\R^n$, $\vecy\in\R^n$, $\hat{\vecy}=\text{proj}_W\vecy$.

Da er $\hat{\vecy}$ punktet på W som ligger nærmest $\vecy$.
$$||\vecy - \hat{\vecy}|| < ||\vecy - \vecv||,
  \qquad \forall\ \vecv\neq\hat{\vecy}\in W$$

      \subsubsection{Teorem 10}
        For en orto\emph{normal} basis $\{ \vecu_1, ..., \vecu_p \}$,
fungerer ortogonal projeksjon likt som vanlig, men litt enklere fordi
lengden av hver vektor er 1.
$$\text{proj}_W\vecy
  = (\vecy\cdot\vecu_1)\vecu_1
    + (\vecy\cdot\vecu_2)\vecu_2
    + ...
    + (\vecy\cdot\vecu_p)\vecu_p$$
Hvis $U = [\vecu_1\ \vecu_2\ ...\ \vecu_p]$ så er
$$\text{proj}_W\vecy = UU^T\vecy, \quad \forall\ \vecy\in\R^n$$

    \subsection{Gram-Schmidt prosessen}
      \subsubsection{Teorem 11}
        En enkel algoritme for å lage ortogonal eller ortonormal basis.

Gitt en basis $\vecx_1, ..., \vecx_p$ for et underrom W.
$$\vecv_1 = \vecx_1$$
$$\vecv_2 = \vecx_2 - \frac{\vecx_2\cdot\vecv_1}{\vecv_1\cdot\vecv_1}\vecv_1$$
$$\vecv_3 = \vecx_3
  - \frac{\vecx_3\cdot\vecv_1}{\vecv_1\cdot\vecv_1}\vecv_1
  - \frac{\vecx_3\cdot\vecv_2}{\vecv_2\cdot\vecv_2}\vecv_2$$
$$\vdots$$
$$\vecv_p = \vecx_p
  - \frac{\vecx_p\cdot\vecv_1}{\vecv_1\cdot\vecv_1}\vecv_1
  - \frac{\vecx_p\cdot\vecv_2}{\vecv_2\cdot\vecv_2}\vecv_2
  - ...
  - \frac{\vecx_p\cdot\vecv_{p-1}}{\vecv_{p-1}\cdot\vecv_{p-1}}\vecv_{p-1}$$
Nå er $\vecv_1, ..., \vecv_p$ en ortogonal basis for W.
$$\Span{\vecv_1, ..., \vecv_p} = \Span{\vecx_1, ..., \vecx_p}$$

      \subsubsection{Teorem 12}
        Hvis A $m\times n$ har lineært uavhengige kolonner,
så kan A faktoriseres
$$A = QR$$
Hvor Q $m\times n$ har kolonner fra en ortonormal basis for $\Col{A}$.
Og R $n\times n$ er øvretriangulær invertibel matrise med positive elementer
på diagonalen.

    \subsection{Minstekvadraters problem}
      \subsubsection{Definisjon - minste kvadraters løsning}
        For A $m\times n$ og $\vecb\in\R^n$.
En minste kvadraters løsning av $A\vecx=\vecb$ er
en $\hat{\vecx}$ s.a.
$$||\vecb-A\hat{\vecx}|| \leq ||\vecb-A\vecx||,
  \qquad \forall\ \vecx \in \R^n$$

      \subsubsection{Teorem 13}
        Mengden av minste kvadraters løsninger av $A\vecx=\vecb$,
sammenfaller med (den ikketomme) mengden av løsninger av normalligningene
$$A^TA\vecx = A^T\vecb$$

      \subsubsection{Teorem 14}
        For A $m\times n$ er følgende ekvivalent
\begin{enumerate}
  \item $A\vecx=\vecb$ har unik minste kvadraters løsning
        $\forall\ \vecb\in\R^m$.
  \item Kolonnene i A er lineært uavhengig.
  \item $A^TA$ er invertibel.
\end{enumerate}
Når disse er sanne, så er minste kvadraters løsning $\hat{\vecx}$ gitt ved
$$\hat{\vecx} = (A^TA)^{-1}A^T\vecb$$

      \subsubsection{Observasjon - alternativ metode}
        For $A\vecx=\vecb$, la $\hat{\vecb} = \text{proj}_{\Col{A}}\vecb$.
Da finner man minste kvadraters løsning ved
$$A\hat{\vecx}=\hat{\vecb}$$

      \subsubsection{Teorem 15}
        For A $m\times n$ med lin. uavh. kolonner, la $A=QR$ som i teorem 12.

Da fins unik minste kvadraters løsning
$$\hat{\vecx} = R^{-1}Q^T\vecb, \qquad \forall\ \vecb\in\R^m$$

    \subsection{Anvendelser til lineære modeller}
      \subsubsection{Begreper}
        Istedenfor å skrive $A\vecx=\vecb$ så skriver vi $X\vecB=\vecy$.

$X$ kalles designmatrisen.

$\vecB$ kalles parametervektoren.

$\vecy$ kalles observasjonsvektoren.

      \subsubsection{Metode - Regresjon}
        Gitt et set datapunkter $(x_1,y_1), ..., (x_m,y_m)$,
så kan man finne en lineærkombinasjon av funksjoner $f_1, ..., f_n$
som best passer punktene.
$$y = \beta_1f_1 + \beta_2f_2 + ... + \beta_nf_n$$
Dette gjøres ved å finne vektene $\beta_1, ..., \beta_n$
fra ligningen $X\vecB=\vecy$.

$$\begin{bmatrix}
    f_1(x_1) & f_2(x_1) & ... & f_n(x_1) \\
    f_1(x_2) & f_2(x_2) & ... & f_n(x_2) \\
             &          & ... &          \\
    f_1(x_m) & f_2(x_m) & ... & f_n(x_m)
  \end{bmatrix}
  \begin{bmatrix}
    \beta_1 \\
    \beta_2 \\
    ...     \\
    \beta_n
  \end{bmatrix}
  =
  \begin{bmatrix}
    y_1 \\
    y_2 \\
    ...     \\
    y_m
  \end{bmatrix}
$$

Funksjonene kan være hva man vil.
F.eks. $f_1(x) = 1$ og $f_2(x) = x$ for en lineær regresjon.

    \subsection{Indreproduktrom}
      \subsubsection{Definisjon - indreprodukt(rom)}
        Et indreprodukt er en funksjon som tar to vektorer $\vecu,\vecv\in V$
og tilordner et reelt tall $\langle\vecu,\vecv\rangle$,
hvor følgende aksiomer er tilfredsstilt.
\begin{enumerate}
  \item $\langle\vecu,\vecv\rangle = \langle\vecv,\vecu\rangle$
  \item $\langle\vecu+\vecv,\vecw\rangle
         = \langle\vecu,\vecw\rangle + \langle\vecv,\vecw\rangle$
  \item $\langle c\vecu,\vecv\rangle = c\langle\vecu,\vecv\rangle$
  \item $\langle\vecu,\vecu\rangle \geq 0, \qquad
         \langle\vecu,\vecu\rangle=0 \iff \vecu=0$
\end{enumerate}

Et vektorrom med et indreprodukt, kalles et indreproduktrom.

      \subsubsection{Begrep - lengde, avstand, ortogonalitet}
        Lengde (norm) til en vektor
$$||\vecv|| = \sqrt{\langle\vecv,\vecv\rangle}$$

Avstand mellom to vektorer
$$||\vecu-\vecv||$$

Ortogonalitet
$$\langle\vecu,\vecv\rangle = 0$$

      \subsubsection{Teorem 16 - Cauchy-Schwartz ulikhet}
        $$|\langle\vecu,\vecv\rangle| \leq ||\vecu||\ ||\vecv||, \qquad
  \forall\ \vecu,\vecv\in V$$

      \subsubsection{Teorem 17 - trekantulikheten}
        $$||\vecu+\vecv|| \leq ||\vecu|| + ||\vecv||$$

      \subsubsection{Definisjon - indreprodukt for $C[a,b]$}
        Vektorrommet $C[a,b]$ er alle funksjoner som er kontinuerlige på
$a \leq t \leq b$.

En vanlig definisjon av indreprodukt er
$$\langle f,g \rangle
  = \int_a^b f(t)g(t)\ dt$$

    \subsection{Anvendelser til indreproduktrom}
      \subsubsection{}
        TODO vektede minste kvadrater, trendanalyse, fourierserie
  \section{Kpt.7 - Symmetriske Matriser og Kvadratisk Form}
    \subsection{Diagonalisering av symmetriske matriser}
      \subsubsection{Begrep - symmetrisk matrise}
        Hvis A er s.a. $A = A^T$, så er den symmetrisk.

      \subsubsection{Teorem 1}
        Hvis A er symmetrisk, så er egenvektorer fra ulike egenrom
ortogonale mot hverandre.

      \subsubsection{Begrep - ortogonalt diagonaliserbar}
        A $n\times n$ er ortogonalt diag.bar hvis det fins:
ortogonal matrise P s.a. $P^{-1} = P^T$, og en diagonal matrise D s.a.
$$A = PDP^{-1} = PDP^T$$

      \subsubsection{Teorem 2}
        A $n\times n$ er ortogonalt diag.bar $\iff$
A er symmetrisk matrise.

      \subsubsection{Teorem 3 - spektralteoremet for symmetriske matriser}
        For A $n\times n$ gjelder:
\begin{enumerate}
  \item A har $n$ reelle egenverdier, med multiplisitet.
  \item Dimensjon til egenrom er lik multiplisiteten til dets egenverdi.
  \item Egenvektorer fra ulike egenrom er ortogonale på hverandre.
  \item A er ortogonalt diagonaliserbar.
\end{enumerate}

      \subsubsection{Observasjon - spektral dekomposisjon}
        $$A = PDP^T
    = \lambda_1\vecu_1\vecu_1^T + ... + \lambda_n\vecu_n\vecu_n^T$$

    \subsection{Kvadratisk form}
      \subsubsection{Definisjon - kvadratisk form}
        En kvadratisk form på $\R^n$ er en funksjon Q s.a.
$$Q(\vecx) = \vecx^T A \vecx$$
Hvor A er en symmetrisk matrise.

      \subsubsection{Metode - koeffisienter fra matrisen}
        Forholdet mellom A og kvadratisk form:
$$A = \begin{bmatrix}
        a   & d/2  & e/2 \\
        d/2 & b    & f/2 \\
        e/2 & f/2  & c
      \end{bmatrix}$$

$$ax_1^2 + bx_2^2 + cx_3^2 + dx_1x_2 + ex_1x_3 + fx_2x_3$$

      \subsubsection{Metode - kryssproduktledd}
        $$Q(\vecx)
  = \vecx^T A \vecx
  = \begin{bmatrix} x_1 & x_2 & x_3 \end{bmatrix}
    \begin{bmatrix} .&.&.\\.&.&.\\.&.&. \end{bmatrix}
    \begin{bmatrix} x_1 \\ x_2 \\ x_3 \end{bmatrix}$$

For å se hvilken koeffisent som går til hvilken $x_jx_k$ kan man se på tabellen:

\begin{tabular}{c|c c c}
     & x1 & x2 & x3 \\ \hline
  x1 & .  & .  & .  \\
  x2 & .  & .  & .  \\
  x3 & .  & .  & .
\end{tabular}

      \subsubsection{Metode - variabelskifte}
        Hvis A ikke er diagonal kan det være lurt med variabelskifte.
$$\vecx = P\vecy, \qquad \vecy = P^{-1} \vecx$$
Her er $\vecy$ koordinatvektor til $\vecx$ for en basis $\B$,
slik som i kpt 4.4.
$$\vecx = P_\B [\vecx]_\B, \qquad P = [\vecb_1 \ ...\  \vecb_n]$$

Kvadratisk form blir nå enklere
$$\vecx^T A \vecx
  = (P\vecy)^T A (P\vecy)
  = \vecy^TP^T A P\vecy
  = \vecy^T (P^T A P) \vecy$$

Men A er symmetrisk så
$$\vecx^T A \vecx = \vecy^T D \vecy$$

      \subsubsection{Teorem 4 - prinsipalakseteoremet}
        Hvis A $n\times n$ er symmetrisk,
så fins et ortogonalt variabelskifte $\vecx = P\vecy$ s.a.
$$\vecx^TA\vecx = \vecy^TD\vecy$$
uten kryssproduktledd.

Kolonnene i P kalles prinsipalaksene til den kvadratiske formen.

      \subsubsection{Observasjon - geometrisk tolkning}
        For invertibel A $n\times n$ og kvadrtisk form $Q(\vecx)=\vecx^TA\vecx$,
kan man velge en konstant $c$ og se på
$$\vecx^TA\vecx = c$$

Det tilsvarer ligningen for enten en hyperbel, ellipse, to kryssende linjer,
et punkt, eller ingen punkter.

      \subsubsection{Definisjon - definit}
        Kvadratisk form Q er:
\begin{enumerate}
  \item Positiv definit hvis $Q(\vecx)>0, \quad \forall \vecx\neq\vecO$.
  \item Negativ definit hvis $Q(\vecx)<0, \quad \forall \vecx\neq\vecO$.
  \item Indefinit hvis $Q(\vecx)$ har både pos. og neg. tall.
\end{enumerate}

      \subsubsection{Teorem 5 - kvadratisk form og egenverdier}
        Kvadratisk form er
\begin{enumerate}
  \item Positiv definit $\iff$ egenverdiene til A er kun positive.
  \item Negativ definit $\iff$ egenverdiene til A er kun negative.
  \item Indefinit $\iff$ A har både pos. og neg. egenverdier.
\end{enumerate}

    \subsection{Begrenset optimalisering}
      \subsubsection{Metode 1 - }
        En vanlig begrensning er $\vecx^T\vecx = 1$.

Anta at Q ikke har kryssproduktledd, og at $a \geq b \geq c$:
$$Q = ax_1^2 + bx_2^2 + cx_3^2
  \leq ax_1^2 + ax_2^2 + ax_3^2
  = a(x_1^2 + x_2^2 + x_3^2)
  = a\cdot\vecx^T\vecx
  = a$$
Så vi fant en maksimumsverdi
$$Q \leq a$$

      \subsubsection{Teorem 6}
        La A være symmetrisk og
$$m = \Min{\vecx^TA\vecx}{||\vecx||=1}$$
$$M = \Max{\vecx^TA\vecx}{||\vecx||=1}$$

Da er $M = \lambda_1$ største egenverdi til A,
og $Q(\vecx)=M$ når $\vecx$ er en enhetsvektor $\vecu_1$ tilsvarende M.

Tilsvarende er $m$ minste egenverdi til A,
og $Q(\vecx)=m$ når $\vecx$ er en enhetsvektor tilsvarende m.

      \subsubsection{Teorem 7}
        La $A, \lambda_1, \vecu_1$ være som i teorem 6.
Maksimum av Q under begrensningene
$$\vecx^T\vecx=1, \quad \vecx^T\vecu_1=0$$
er den nest største egenverdien $\lambda_2$.
Den verdien fåes når $\vecx=\vecu_2$ egenvektor tilsvarende $\lambda_2$.

      \subsubsection{Teorem 8}
        A $n\times n$,
ortog.diag. $A=PDP^{-1}$,
$D=\text{diag}(\lambda_1\geq ...\geq\lambda_n)$,
kolonner i P er tilsvarende enhetsegenvektorer $\vecu_1, ..., \vecu_n$.

Da vil max av $Q$ under begrensningene være
$$\vecx^T\vecx=1, \quad \vecx^T\vecu_1=0,
  \quad ..., \quad \vecx^T\vecu_{k-1}=0$$
egenverdi $\lambda_k$, og maks gitt ved $\vecx = \vecu_k$.

    \subsection{Singulærverdidekomposisjon}
      \subsubsection{Begrep - singulærverdier}
        For en ikkekvadratisk A $m\times n$, så kan vi regnu ut $A^TA$.

Videre kan vi regne ut egenverdiene $\lambda_i$ til $A^TA$.

Singulærverdiene $\sigma$ til A er da
$$\sigma_i = \sqrt{\lambda_i}$$

Vanligvis sorteser de s.a.
$$\sigma_1 \geq \sigma_2 \geq ... \geq \sigma_n$$

      \subsubsection{Teorem 9}
        Hvis egenvektorene til $A^TA$ kan gjøres til en
ortonormal basis $\vecv_1,...,\vecv_n$,
ordnet slik at $\lambda_1 \geq ... \geq \lambda_n$.
Og hvis A har $r$ ikkenull singulærverdier.

Så er $\{ A\vecv_1, ..., A\vecv_r \}$
en ortogonal basis for $\Col{A}$, og $\Rank{A}=r$.

      \subsubsection{Teorem 10 - singulærverdidekomposisjon}
        La A være $m\times n$ med rang r.

Da fins det en $\Sigma$ $m\times n$ på formen
$$\Sigma = \begin{bmatrix} D & 0 \\ 0 & 0 \end{bmatrix}$$
hvor $D = \text{diag}(\sigma_1 \geq ... \geq \sigma_r \geq 0)$.

Og det fins også en ortogonal $U$ $m\times m$,
og en ortogonal $V$ $n\times n$ s.a.
$$A = U\Sigma V^T$$

U og V er ikke unikt bestemt.

      \subsubsection{Metode - singulærverdidekomposisjon}
        \paragraph{Ortogonalt diagonaliser $A^TA$} \hfill \\
Finn egenverdiene og tilsvarende ortonormal mengde egenvektorer.

\paragraph{Konstruer V} \hfill \\
Sorter egenverdiene til $A^TA$ i synkende rekkefølge,
og bruk tilsvarende egenvektorer som kolonner i V
$$V = [\ \vecv_1\ \vecv_2\ ...\  \vecv_n \ ]$$

\paragraph{Konstruer $\Sigma$} \hfill \\
Bruk singulærverdiene og la
$D = \text{diag}(\sigma_1 \geq ... \geq \sigma_r \geq 0)$.
Konstruer $\Sigma$ til å være $m\times n$
$$\Sigma = \begin{bmatrix} D & 0 \\ 0 & 0 \end{bmatrix}$$

\paragraph{Konstruer U} \hfill \\
Enten kan man bruke matrisemultiplikasjon og inverse for å finne U,
eller så kan man bruke
$$U = [\ \vecu_1\ \vecu_2\ ...\  \vecu_n \ ]$$
$$\vecu_i = \frac{1}{\sigma_i}A\vecv_i$$

      \subsubsection{Teorem - IMT konkludert}
        For A $n\times n$ er følgende ekvivalent
\begin{enumerate}
  \item A er invertibel.
  \item $(\Col{A})^\perp = \{\vecO\}$
  \item $(\Nul{A})^\perp = \R^n$
  \item $\Row{A} = R^n$
  \item A har $n$ ikkenull singulærverdier.
\end{enumerate}

      \subsubsection{}
        TODO condition number, basis for fundamentale underrom,
             redusert SVD, pseudoinvers av A, minste kvadrater.
    \subsection{Ikke pensum?}
      TODO Ikke pensum?
  \section{Notat 1}
    \subsubsection{}
      TODO
  \section{Notat 2}
    \subsubsection{}
      TODO
  TODO egenfunksjon
\end{document}
