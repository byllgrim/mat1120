Et vektorrom er en ikketom mengde V.
Den består av såkalte \emph{vektorer}.
Disse vektorene må være beskrevet av 2 operasjoner:
Addisjon og skalarmultiplikasjon.

De to operasjonene defineres av følgende aksiomer:
La $\vecu,\vecv,\vecw \in V$
\begin{enumerate}
  \item $\vecu + \vecv \in V$
  \item $\vecu + \vecv = \vecv + \vecu$
  \item $(\vecu+\vecv)+\vecw = \vecu+(\vecv+\vecw)$
  \item $\exists \vecO \in V s.a. \vecu+\vecO=\vecu$
  \item $\forall \vecu \in V,\ \exists -\vecu \in V s.a. \vecu+(-\vecu)=\vecO$
  \item $c\vecu \in V,\ c\in\mathbb{R}$
  \item $c(\vecu+\vecv) = c\vecu+c\vecv$
  \item $(c+d)\vecu = c\vecu+d\vecu$
  \item $c(d\vecu) = (cd)\vecu$
  \item $1\vecu = \vecu$
\end{enumerate}
